\section{Resource Access Protocol}

Now we will discus 4 main resource access protocol for periodic task which are:

\begin{itemize}
\item Non-Preemptive Protocol
\item Highest Locker Priority
\item Priority Inheritance Protocol
\item Priority Ceiling Protocol
\end{itemize}


The first protocol is for non-preemptive-able schedule the the last three are for . The following list are the important notation that I use in explaining each protocol. The main idea about those protocol is to avoid priority invention, so that the blocking time experienced by the task that have highest priority by lower priority task could be reduced. Before I explaining the protocols, here are some notation that need to be understood.

\begin{itemize}
\item $ B_{i} $ denotes the maximum blocking time task $ \tau_{i} $ can experience.
\item $ z_{i,k} $ denotes a generic critical section of task $ \tau_{i} $ guarded by semaphore $ S_{k} $.
\item  $ Z_{i,k} $ denotes the longest critical section of task $ \tau_{i} $ guarded by semaphore $ S_{k} $.
\item $ \delta_{i,k} $ denotes the duration of $ Z_{i,k} $.
\item $ z_{i,h} \subset z_{i,k} $ indicates that $ z_{i,h} $ is entirely contained in $ z_{i,k} $.
\item $ \sigma_{i} $ denotes the set of semaphores used by $ \tau_{i} $.
\item $ \sigma_{i,j} $ denotes the set of semaphores that can block $ \tau_{i} $, used by the lower-priority task $ \tau_{j} $ .
\item $ \gamma_{i,j} $ denotes the set of the longest critical sections that can block $ \tau_{i} $, accessed by the lower priority task $ \tau_{j} $ . That is,
\begin{center}
$ \gamma_{i}=\{Z_{j,k} | P_{j}<P_{i}$ and $(S_{k}\geq \sigma_{i,j}) \} $
\end{center}
\item $ \gamma_{i}$ denotes the set of all the longest critical sections that can block $ \tau_{i} $. That is,
\begin{center}
$ \sigma_{i} =\underset{j:P_{j}>P_{i}}{\cup} \gamma_{i,j}$
\end{center}
\item
\end{itemize}




\section{Non-Preemptive Protocol}

\subsection{Definition}

This protocol is a simple protocol and named non preemptive because it avoid any interruption on running task $\tau_{j}$ that accessing a resource $ R_{k} $that guarded by , $ S_{k} $. To reduce total blocking time experienced by the task $\tau_{i}$ that have highest priority, this protocol just increase the priority of the task $\tau_{j}$ that currently accessing the resource $\tau_{j}$, so that the task will not be interrupted and can be done much faster.Without this protocol, the task that highest priority $\tau_{i}$ will interrupt the task $\tau_{j}$ that currently accessing the resource $ R_{k} $ even though the task cannot access the resource because it already guarded by $ S_{k} $. The scheduler then switch back to the task before to finish its process, this switch context process could cause longer blocking time experienced by the highest priority task. After the task$\tau_{j}$ finish accessing the resources, its priority will be back to its nominal priority $ P_{j} $.These situations can be compared through figure \ref{fig:An_example_of_priority_inversion} and \ref{fig:Example_of_NPP_preventing_priority_inversion} . So, the priority of the task $\tau_{i}$that currently accessing the resource is 

\begin{center}
$p_{i}(R_{k})=\underset{h}{\mathrm{max}} \{P_{h}\}  $

\end{center}

\begin{figure}[h]
    \centering
    \includegraphics[width=0.5\textwidth]{An_example_of_priority_inversion}
    \caption{An example of priority inversion.. \cite{b5}}
    \label{fig:An_example_of_priority_inversion}
\end{figure}

\begin{figure}[h]
    \centering
    \includegraphics[width=0.5\textwidth]{Example_of_NPP_preventing_priority_inversion}
    \caption{Example of NPP preventing priority inversion. \cite{b5}}
    \label{fig:Example_of_NPP_preventing_priority_inversion}
\end{figure}


\subsection{Blocking time computation}

	The total of critical section of lower priority task $\tau_{j}$ blocking higher priority task $\tau_{i}$ is

\begin{center}
$ \gamma_{i}=\{Z_{j,k} | P_{j}<P_{i}, k=1,...,m \} $ \cite{b5}
\end{center}

Hence the total duration highest priority task is blocked is

\begin{center}

$B_{i}(R_{k})=\underset{j,k}{\mathrm{max}} \{ \delta_{j,k}-1 | Z_{j,k} \in \gamma_{i}\}  $ \cite{b5}
\end{center}

\subsection{Implementation Strategies}

As state by \cite{b6} - All commercial RTOSs have a means for beginning and ending a critical section. Invoking this Scheduler 
operation prevents all task switching from occurring during the critical section. If we write our own RTOS, the most common way to do this is to set the Disable Interrupts bit on our processor's flags register. The precise details of this vary, naturally, depending on the specific processor.

\subsection{Sample Model}

The following example model is totally taken from \cite{b6}.

An example of the use of this pattern is shown in Figure \ref{fig:npp_model}. This example contains three tasks: Device Test (highest priority), Motor Control (medium priority), and Data Processing (lowest priority). Device Test and Data Processing share a resource called Sensor, whereas Motor Control has its own resource called Motor. 

The scenario starts off with the lowest-priority task, Data Processing, accessing the resource that starts up a critical section. During this critical section both the Motor Control task and the Device Test task become ready to run but cannot because task switching is disabled. When the call to the resource is almost done, the Sensor.gimme() operation makes a call to the scheduler to end the critical section. The scenario shows three critical sections, one for each of the running tasks. Finally, at the end, the lowest-priority task is allowed to complete its work and then returns to its Idle state.

\begin{figure}[h]
    \centering
    \includegraphics[width=0.5\textwidth]{npp_model}
    \caption{Sample Model for Non-Preemptive Protocol \cite{b6}}
    \label{fig:npp_model}
\end{figure}

\subsection{Problem Arise}

As shown in figure \ref{fig:Example_in_which_NPP_causes_unnecessary_blocking_on_T1}, this protocol will block highest priority task $ \tau_{1} $ even though the task will not access the resource.This problem could be solved in the next protocol which is Highest Locker Priority (HLP) protocol.

\begin{figure}[h]
    \centering
    \includegraphics[width=0.5\textwidth]{Example_in_which_NPP_causes_unnecessary_blocking_on_T1}
    \caption{Example in which NPP causes unnecessary blocking on $ \tau_{1} $ \cite{b5}}
    \label{fig:Example_in_which_NPP_causes_unnecessary_blocking_on_T1}
\end{figure}



 

\section{Highest Locker Priority}

\subsection{Definition}

 Highest Locker Priority (HLP) is the improvement of the previous protocol to allow the highest priority task $\tau_{i}$ that doesn't use resource $R_{k}$ to interrupt the lower priority task $\tau_{j}$ that use the resource, $R_{k}$ by limiting the raised priority of task $\tau_{j}$. So,
 
\begin{center}
 $p_{i}(R_{k})=\underset{h}{\mathrm{max}} \{P_{h}| \tau_{h}$ uses $R_{k}\}  $ 
\end{center}

This dynamic priority then set back to its nominal value $P_{i}$ when the task leave its critical section. The maximum raised priority of task $\tau_{j}$ is called priority ceiling $ C(R_{k}) $ and computed off-line.  The maximum priority $ C(R_{k}) $ of the tasks sharing $ R_{k} $ is the computed online such

\begin{center}
$C(R_{k})\stackrel{def}{=}\underset{h}{\mathrm{max}} \{P_{h}| \tau_{h}$ uses $R_{k}\}  $
\end{center}

Since the priority of lower priority task $\tau_{j}$ is raised as soon as the task entering $ R_{k} $, this protocol also known as Immediate Priority Ceiling. This protocol can be visualize as in \ref{fig: Example_of_schedule_under_HLP} where task $ \tau_{1} $ have higest priority and task $ \tau_{3} $ is the first task arrive

\begin{figure}[h]
    \centering
    \includegraphics[width=0.5\textwidth]{Example_of_schedule_under_HLP}
    \caption{ Example of schedule under HLP, where $ p3 $ is raised at the level $ C(R) = P_{2} $ as soon as $ \tau_{3} $ starts using resource R \cite{b5}}
    \label{fig:Example_of_schedule_under_HLP}
\end{figure}

 
\subsection{Blocking Time Computation}

So, total of critical section of lower priority task $\tau_{j}$ blocking higher priority task $\tau_{i}$ is reduced by adding new parameter as shown below.

\begin{center}
$ \gamma_{i}=\{Z_{j,k} | P_{j}<P_{i} $ and $ C(R_{k})\geq P_{i} \} $
\end{center}

According to \cite{b5} - Under HLP, a task $ \tau_{i} $ can be blocked, at most, for the duration of a single critical section belonging to the set $ \gamma_{i} $ and this theorem is proved by contradiction, assuming that $ \tau_{i} $ is blocked by two critical sections, $ z_{1,a} $ and $ z_{2,b} $. For this to happen, both critical sections must belong to different tasks ($ \tau_{1} $ and $ \tau_{2} $) with priority lower than $ P_{i} $, and both must have a ceiling higher than or equal to $ P_{i} $. That is, by assumption, we must have

\begin{center}
$ P_{1}< P_{i}\leq C(R_{a})$

$ P_{2}< P_{i}\leq C(R_{b})$
\end{center}

Now, $ \tau_{i} $  can be blocked twice only if $ \tau_{1} $  and $ \tau_{2} $  are both inside the resources when $ \tau_{i} $ arrives, and this can occur only if one of them (say $ \tau_{1} $ ) preempted the other inside the critical section. But, if $ \tau_{1} $  preempted $ \tau_{2} $ inside $ z_{2,b} $  it means that $ P_{2}> C(R_{b}) $ , which is a contradiction. Hence, the theorem follows.

As shown in figure \ref{fig:Example_of_schedule_under_HLP}, $ \tau_{i} $ can be block at maximum once, means that

\begin{center}
$B_{i}(R_{k})=\underset{j,k}{\mathrm{max}} \{ \delta_{j,k}-1 | Z_{j,k} \in \gamma_{i}\}  $
\end{center}

We need to minus one unit of time because the lower priority task $ \tau_{j} $ need to access $ R_{k} $ atleast one unit time earlier then $ \tau_{i} $ to block it.

\subsection{Problem Arise}

Despite the fact that this algorithm improve the previous algorithm, it still could produce some unnecessary blocking. This algorithm block a task at the time it attempt, before it actually require a resource \cite{b5}. It also says that - If a critical section is contained only in one branch of a conditional statement, then the task could be unnecessarily blocked, since during execution it could take the branch without the resource.

This protocol was improved by Priority Inheritance protocol that will be expalain in the next section.
 











\section{Priority Inheritance Protocol}

\subsection{Definition}

$C(R_{k})\stackrel{def}{=}\underset{h}{\mathrm{max}} \{P_{h}| \tau_{h} uses R_{k}\}  $

\ref{fig:pip}
\begin{figure}[h]
    \centering
    \includegraphics[width=0.5\textwidth]{pip}
    \caption{Example of Priority Inheritance Protocol. \cite{b5}}
    \label{fig:pip}
\end{figure}

\ref{fig:pip_nested}
\begin{figure}[h]
    \centering
    \includegraphics[width=0.5\textwidth]{pip_nested}
    \caption{Priority inheritance with nested critical sections.\cite{b5}}
    \label{fig:pip_nested}
\end{figure}

\ref{fig:tpi}
\begin{figure}[h]
    \centering
    \includegraphics[width=0.5\textwidth]{tpi}
    \caption{Example of transitive priority inheritance.\cite{b5}}
    \label{fig:tpi}
\end{figure}

\subsection{Blocking Time computation}

$ \sigma \frac{dir}{i,j} =  \sigma_{i} \cap \sigma_{j} $

$ \sigma \frac{pt}{i,j} =  \underset{h:P_{h}>P_{i}}{\cup} \sigma_{h} \cap \sigma_{j} $


 $ \sigma_{i,j} = \sigma \frac{dir}{i,j}\cup\sigma \frac{pt}{i,j} =  \underset{h:P_{h}>P_{i}}{\cup} \sigma_{h} \cap \sigma_{j} $

$ \gamma_{i,j}=\{Z_{j,k} | P_{j}<P_{i} and R_{k}\in \sigma_{i,j} \} $

$ \sigma_{i} =\underset{j:P_{j}>P_{i}}{\cup} \gamma_{i,j}$


$ B \frac{l}{i} =\sum_{j=i+1}^{n} \underset{k}{\mathrm{max}} \{\sigma_{j,k}-1| C(S_{k})\geq P_{i}\}$

$ B_{i} =\underset{j,k}{\mathrm{max}} \{\delta_{j,k}-1| Z_{j,k}\in \gamma_{i}\}$

$ \gamma_{1}=\{\} $

\subsection{Problem Arise}


\subsubsection{Chained Blocking}

\ref{fig:Example_of_chained_blocking}
\begin{figure}[h]
    \centering
    \includegraphics[width=0.5\textwidth]{Example_of_chained_blocking}
    \caption{Example of chained blocking.\cite{b5}}
    \label{fig:Example_of_chained_blocking}
\end{figure}

\subsubsection{Deadlock}

\ref{fig:Example_of_deadlock}
\begin{figure}[h]
    \centering
    \includegraphics[width=0.5\textwidth]{Example_of_deadlock}
    \caption{Example of deadlock.\cite{b5}}
    \label{fig:Example_of_deadlock}
\end{figure}


















\section{Priority Ceiling Protocol}

$ C(S_{k})\stackrel{def}{=}\underset{i}{\mathrm{max}} \{P_{i}| S_{k} in \sigma_{i}\}   $

$ p_{j}(R_{k})=max\{P_{j},\underset{h}{\mathrm{max}}\{P_{h}|\tau_{h} is blocked on R_{k}\}\} $

$ C(S_{A})=P_{1} $

$ P_{0}\geq P_{i} >C^{\ast} $

$ \gamma_{i}=\{Z_{j,k} | (P_{j}<P_{i}) and C(R_{k})\geq P_{i} \} $