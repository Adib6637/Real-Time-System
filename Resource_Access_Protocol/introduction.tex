\section{Introduction}

In this paper, two resource access protocols that are  used in real-time systems will be discussed. But, before we get into the resource access protocol, we must first comprehend why it exists. Hence, it is necessary to describe it, and I will begin with the definition of the real-time system.\par 

According to \cite{b1} - "Real Time System is a system whose response time and delay time are deterministic without uncertainty and non-reproducibility and has an internal configuration that makes the worst value predictable or makes it easy to produce an educated guess value".\par

As state by \cite{b2} - "Real-time systems are designed to perform tasks that must be executed within precise cycle deadlines (down to microseconds)."

So, one of the most important key factor of a real-time system is its predictability, which ensures that it can give the appropriate output at the right moment. If two or more tasks use the same resource with a defined synchronization mechanism, as indicated in the \textit{background} and \textit{problem description} section, this will become a major issue. This is where the resource access protocol comes in to assist us in resolving the issue. \par

In the following section, I will go through the basics of the concepts and terms used in this paper so that each protocol's explanation is clear. Then we will go through the major issue with real-time systems with tasks sharing a resource. Following that, the two resource access protocols will be discussed.\par
 





