\section{Problem Definition: Invention Priority}

In theory, a set of tasks that are schedulable will be executed upon on its arriving time and preempted if a task with higher priority arrives. That will not be always true if we apply resource constraint. The problem arrives when a scheduler wants to preempt a lower task that is currently accessing a  resource, R and wants to execute a higher priority task that also wants to use the resource. Since the lower priority task did not yet produce signal(S) primitive where S is the binary semaphore, the higher priority task will be blocked. This phenomenon called priority invention. The implication of this phenomenon is that it will cause unbounded delay on execution of task with higher priority and reduce the predictability of the system because the highest priority task could miss its deadline. The priority invention is illustrated in ``Fig. \ref{fig:Example_in_which_NPP_causes_unnecessary_blocking_on_T1}'' where $ \tau_{1} $ have the highest priority.

Here is where we need resource access protocol to make some adjustments to the scheduler so that the implication of invention priority could be reduced. We are going to discuss this solution in the next sections.